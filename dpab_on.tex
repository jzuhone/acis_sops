\documentclass[11pt]{article}

\usepackage{lscape,color}

\topmargin -0.75truein
\oddsidemargin -0.4truein
\textheight 9.25truein
\textwidth 6.7truein
\hbadness=10001
\hfuzz=200pt


\begin{document}
%\input dspace12.tex
%\input pstricks.tex
%\input psfig
\newcommand{\be}{\begin{enumerate}}
\newcommand{\ee}{\end{enumerate}}
\newcommand{\bc}{\begin{center}}
\newcommand{\ec}{\end{center}}
\newcommand{\bi}{\begin{itemize}}
\newcommand{\ei}{\end{itemize}}
\newcommand{\bd}{\begin{description}}
\newcommand{\ed}{\end{description}}
\newcommand{\bt}{\begin{tabbing}}
\newcommand{\et}{\end{tabbing}}
\newcommand{\eg}{{\it e.g.~}}
\newcommand{\ie}{{\it i.e.~}}
\newcommand{\ul}{\underline}
\newcommand{\axaf}{{\em AXAF}}
\def\la{\hbox{\rlap{$<$}\lower0.5ex\hbox{$\sim$}\ }}


\large
%\vspace*{-0.5in}
\centerline {\bf 4.4\_V2.1 TURN ON DPA B (realtime version) }
\vspace{0.25in}

\normalsize
\noindent{\it Last Revised: March 25, 2016}\\
\noindent{\bf Filename: dpab\_on} \\

\noindent {\bf BRIEF FUNCTIONAL DESCRIPTION:} \\
\normalsize
This is an ``atomic'' procedure which simply powers up the DPA side B.
It should be safe to execute under any
condition except a spacecraft power or thermal emergency.
The telemetry verifiers for the ``Enable'' and ``On''  can be
confusing depending on if either side of the DPA was on before the
procedure started.  If both sides of the DPA were off, the Enable and
On will read Enabled and On even though both sides of the DPA are off.
Once the the On command has been executed the Enable and On will still
read Enabled and On.  The only sure way to tell is to check the input
current and the DPA 5V.  If side A of the DPA happened to be on, then
the Enable and On would correctly read disabled and off before this
procedure is run. This procedure does not assume anything about the
state of DPA A or which BEP is currently selected.  Therefore, it does
not verify that BEP B boots.


\vspace{0.25in}
\noindent The sequence of actions will be:
\be
\item enable and turn on DPA power supply side B
\ee


\vspace{0.15in}
\normalsize
\noindent {\bf ASSUMED INSTRUMENT STATE:} \\
\normalsize
Assumes that the PSMC has power from the spacecraft. \\


%\vspace{0.25in}
\normalsize
\noindent {\bf SPECIAL INITIAL CONDITIONS:} \\
\normalsize
%The environment must be clean enough to allow opening of the valve. \\
%Assumes that \axaf\/ ISIM RCTU is powered on and in telemetry format 6. \\

%\newpage

%\vspace{0.25in}
\normalsize
\noindent {\bf OPERATIONAL CONSTRAINTS/CAUTIONS:} \\
\normalsize



\vspace{0.15in}
\normalsize
\noindent {\bf REFERENCES:} \\
\normalsize

\normalsize
\noindent {\bf CHANGE HISTORY:} \\
\normalsize

{\bf V1.2}
\begin{itemize}
\item changed filenames from ``turnon\_dpab'' to
``dpab\_on''
\item added text to explain the confusion with the logical verifiers
the DPAs
\end{itemize}

{\bf V1.3}
\begin{itemize}
\item changed primary verifier to be the DPA +5~V supply
\item change TLM FMT to 1,2,4or6
\end{itemize}

{\bf V2.0}
\begin{itemize}
\item ACIS Team signed-off version, identical to previous version 1.3
\end{itemize}

{\bf V2.1}
\begin{itemize}
\item Update expected 1DP28BVO range
\item Changed formatting of ``Tlm Fmt'' in table
\item Changed time column from units of seconds to minutes in table
\item Changed text in table column ``Description''
\end{itemize}

\newpage\
\vspace{0.4\textheight}
\bc This page is intentionally blank \ec

\newcommand{\tablecaptiontext}{TURN ON DPA B (realtime version)}
\documentclass[11pt]{article}

\usepackage{lscape,color}

\topmargin -0.75truein
\oddsidemargin -0.4truein
\textheight 9.25truein
\textwidth 6.7truein
\hbadness=10001
\hfuzz=200pt


\begin{document}
%\input dspace12.tex
%\input pstricks.tex
%\input psfig
\newcommand{\be}{\begin{enumerate}}
\newcommand{\ee}{\end{enumerate}}
\newcommand{\bc}{\begin{center}}
\newcommand{\ec}{\end{center}}
\newcommand{\bi}{\begin{itemize}}
\newcommand{\ei}{\end{itemize}}
\newcommand{\bd}{\begin{description}}
\newcommand{\ed}{\end{description}}
\newcommand{\bt}{\begin{tabbing}}
\newcommand{\et}{\end{tabbing}}
\newcommand{\eg}{{\it e.g.~}}
\newcommand{\ie}{{\it i.e.~}}
\newcommand{\ul}{\underline}
\newcommand{\axaf}{{\em AXAF}}
\def\la{\hbox{\rlap{$<$}\lower0.5ex\hbox{$\sim$}\ }}


\large
%\vspace*{-0.5in}
\centerline {\bf 4.4\_V2.1 TURN ON DPA B (realtime version) }
\vspace{0.25in}

\normalsize
\noindent{\it Last Revised: March 25, 2016}\\
\noindent{\bf Filename: dpab\_on} \\

\noindent {\bf BRIEF FUNCTIONAL DESCRIPTION:} \\
\normalsize
This is an ``atomic'' procedure which simply powers up the DPA side B.
It should be safe to execute under any
condition except a spacecraft power or thermal emergency.
The telemetry verifiers for the ``Enable'' and ``On''  can be
confusing depending on if either side of the DPA was on before the
procedure started.  If both sides of the DPA were off, the Enable and
On will read Enabled and On even though both sides of the DPA are off.
Once the the On command has been executed the Enable and On will still
read Enabled and On.  The only sure way to tell is to check the input
current and the DPA 5V.  If side A of the DPA happened to be on, then
the Enable and On would correctly read disabled and off before this
procedure is run. This procedure does not assume anything about the
state of DPA A or which BEP is currently selected.  Therefore, it does
not verify that BEP B boots.


\vspace{0.25in}
\noindent The sequence of actions will be:
\be
\item enable and turn on DPA power supply side B
\ee


\vspace{0.15in}
\normalsize
\noindent {\bf ASSUMED INSTRUMENT STATE:} \\
\normalsize
Assumes that the PSMC has power from the spacecraft. \\


%\vspace{0.25in}
\normalsize
\noindent {\bf SPECIAL INITIAL CONDITIONS:} \\
\normalsize
%The environment must be clean enough to allow opening of the valve. \\
%Assumes that \axaf\/ ISIM RCTU is powered on and in telemetry format 6. \\

%\newpage

%\vspace{0.25in}
\normalsize
\noindent {\bf OPERATIONAL CONSTRAINTS/CAUTIONS:} \\
\normalsize



\vspace{0.15in}
\normalsize
\noindent {\bf REFERENCES:} \\
\normalsize

\normalsize
\noindent {\bf CHANGE HISTORY:} \\
\normalsize

{\bf V1.2}
\begin{itemize}
\item changed filenames from ``turnon\_dpab'' to
``dpab\_on''
\item added text to explain the confusion with the logical verifiers
the DPAs
\end{itemize}

{\bf V1.3}
\begin{itemize}
\item changed primary verifier to be the DPA +5~V supply
\item change TLM FMT to 1,2,4or6
\end{itemize}

{\bf V2.0}
\begin{itemize}
\item ACIS Team signed-off version, identical to previous version 1.3
\end{itemize}

{\bf V2.1}
\begin{itemize}
\item Update expected 1DP28BVO range
\item Changed formatting of ``Tlm Fmt'' in table
\item Changed time column from units of seconds to minutes in table
\item Changed text in table column ``Description''
\end{itemize}

\newpage\
\vspace{0.4\textheight}
\bc This page is intentionally blank \ec

\newcommand{\tablecaptiontext}{TURN ON DPA B (realtime version)}
\documentclass[11pt]{article}

\usepackage{lscape,color}

\topmargin -0.75truein
\oddsidemargin -0.4truein
\textheight 9.25truein
\textwidth 6.7truein
\hbadness=10001
\hfuzz=200pt


\begin{document}
%\input dspace12.tex
%\input pstricks.tex
%\input psfig
\newcommand{\be}{\begin{enumerate}}
\newcommand{\ee}{\end{enumerate}}
\newcommand{\bc}{\begin{center}}
\newcommand{\ec}{\end{center}}
\newcommand{\bi}{\begin{itemize}}
\newcommand{\ei}{\end{itemize}}
\newcommand{\bd}{\begin{description}}
\newcommand{\ed}{\end{description}}
\newcommand{\bt}{\begin{tabbing}}
\newcommand{\et}{\end{tabbing}}
\newcommand{\eg}{{\it e.g.~}}
\newcommand{\ie}{{\it i.e.~}}
\newcommand{\ul}{\underline}
\newcommand{\axaf}{{\em AXAF}}
\def\la{\hbox{\rlap{$<$}\lower0.5ex\hbox{$\sim$}\ }}


\large
%\vspace*{-0.5in}
\centerline {\bf 4.4\_V2.1 TURN ON DPA B (realtime version) }
\vspace{0.25in}

\normalsize
\noindent{\it Last Revised: March 25, 2016}\\
\noindent{\bf Filename: dpab\_on} \\

\noindent {\bf BRIEF FUNCTIONAL DESCRIPTION:} \\
\normalsize
This is an ``atomic'' procedure which simply powers up the DPA side B.
It should be safe to execute under any
condition except a spacecraft power or thermal emergency.
The telemetry verifiers for the ``Enable'' and ``On''  can be
confusing depending on if either side of the DPA was on before the
procedure started.  If both sides of the DPA were off, the Enable and
On will read Enabled and On even though both sides of the DPA are off.
Once the the On command has been executed the Enable and On will still
read Enabled and On.  The only sure way to tell is to check the input
current and the DPA 5V.  If side A of the DPA happened to be on, then
the Enable and On would correctly read disabled and off before this
procedure is run. This procedure does not assume anything about the
state of DPA A or which BEP is currently selected.  Therefore, it does
not verify that BEP B boots.


\vspace{0.25in}
\noindent The sequence of actions will be:
\be
\item enable and turn on DPA power supply side B
\ee


\vspace{0.15in}
\normalsize
\noindent {\bf ASSUMED INSTRUMENT STATE:} \\
\normalsize
Assumes that the PSMC has power from the spacecraft. \\


%\vspace{0.25in}
\normalsize
\noindent {\bf SPECIAL INITIAL CONDITIONS:} \\
\normalsize
%The environment must be clean enough to allow opening of the valve. \\
%Assumes that \axaf\/ ISIM RCTU is powered on and in telemetry format 6. \\

%\newpage

%\vspace{0.25in}
\normalsize
\noindent {\bf OPERATIONAL CONSTRAINTS/CAUTIONS:} \\
\normalsize



\vspace{0.15in}
\normalsize
\noindent {\bf REFERENCES:} \\
\normalsize

\normalsize
\noindent {\bf CHANGE HISTORY:} \\
\normalsize

{\bf V1.2}
\begin{itemize}
\item changed filenames from ``turnon\_dpab'' to
``dpab\_on''
\item added text to explain the confusion with the logical verifiers
the DPAs
\end{itemize}

{\bf V1.3}
\begin{itemize}
\item changed primary verifier to be the DPA +5~V supply
\item change TLM FMT to 1,2,4or6
\end{itemize}

{\bf V2.0}
\begin{itemize}
\item ACIS Team signed-off version, identical to previous version 1.3
\end{itemize}

{\bf V2.1}
\begin{itemize}
\item Update expected 1DP28BVO range
\item Changed formatting of ``Tlm Fmt'' in table
\item Changed time column from units of seconds to minutes in table
\item Changed text in table column ``Description''
\end{itemize}

\newpage\
\vspace{0.4\textheight}
\bc This page is intentionally blank \ec

\newcommand{\tablecaptiontext}{TURN ON DPA B (realtime version)}
\documentclass[11pt]{article}

\usepackage{lscape,color}

\topmargin -0.75truein
\oddsidemargin -0.4truein
\textheight 9.25truein
\textwidth 6.7truein
\hbadness=10001
\hfuzz=200pt


\begin{document}
%\input dspace12.tex
%\input pstricks.tex
%\input psfig
\newcommand{\be}{\begin{enumerate}}
\newcommand{\ee}{\end{enumerate}}
\newcommand{\bc}{\begin{center}}
\newcommand{\ec}{\end{center}}
\newcommand{\bi}{\begin{itemize}}
\newcommand{\ei}{\end{itemize}}
\newcommand{\bd}{\begin{description}}
\newcommand{\ed}{\end{description}}
\newcommand{\bt}{\begin{tabbing}}
\newcommand{\et}{\end{tabbing}}
\newcommand{\eg}{{\it e.g.~}}
\newcommand{\ie}{{\it i.e.~}}
\newcommand{\ul}{\underline}
\newcommand{\axaf}{{\em AXAF}}
\def\la{\hbox{\rlap{$<$}\lower0.5ex\hbox{$\sim$}\ }}


\large
%\vspace*{-0.5in}
\centerline {\bf 4.4\_V2.1 TURN ON DPA B (realtime version) }
\vspace{0.25in}

\normalsize
\noindent{\it Last Revised: March 25, 2016}\\
\noindent{\bf Filename: dpab\_on} \\

\noindent {\bf BRIEF FUNCTIONAL DESCRIPTION:} \\
\normalsize
This is an ``atomic'' procedure which simply powers up the DPA side B.
It should be safe to execute under any
condition except a spacecraft power or thermal emergency.
The telemetry verifiers for the ``Enable'' and ``On''  can be
confusing depending on if either side of the DPA was on before the
procedure started.  If both sides of the DPA were off, the Enable and
On will read Enabled and On even though both sides of the DPA are off.
Once the the On command has been executed the Enable and On will still
read Enabled and On.  The only sure way to tell is to check the input
current and the DPA 5V.  If side A of the DPA happened to be on, then
the Enable and On would correctly read disabled and off before this
procedure is run. This procedure does not assume anything about the
state of DPA A or which BEP is currently selected.  Therefore, it does
not verify that BEP B boots.


\vspace{0.25in}
\noindent The sequence of actions will be:
\be
\item enable and turn on DPA power supply side B
\ee


\vspace{0.15in}
\normalsize
\noindent {\bf ASSUMED INSTRUMENT STATE:} \\
\normalsize
Assumes that the PSMC has power from the spacecraft. \\


%\vspace{0.25in}
\normalsize
\noindent {\bf SPECIAL INITIAL CONDITIONS:} \\
\normalsize
%The environment must be clean enough to allow opening of the valve. \\
%Assumes that \axaf\/ ISIM RCTU is powered on and in telemetry format 6. \\

%\newpage

%\vspace{0.25in}
\normalsize
\noindent {\bf OPERATIONAL CONSTRAINTS/CAUTIONS:} \\
\normalsize



\vspace{0.15in}
\normalsize
\noindent {\bf REFERENCES:} \\
\normalsize

\normalsize
\noindent {\bf CHANGE HISTORY:} \\
\normalsize

{\bf V1.2}
\begin{itemize}
\item changed filenames from ``turnon\_dpab'' to
``dpab\_on''
\item added text to explain the confusion with the logical verifiers
the DPAs
\end{itemize}

{\bf V1.3}
\begin{itemize}
\item changed primary verifier to be the DPA +5~V supply
\item change TLM FMT to 1,2,4or6
\end{itemize}

{\bf V2.0}
\begin{itemize}
\item ACIS Team signed-off version, identical to previous version 1.3
\end{itemize}

{\bf V2.1}
\begin{itemize}
\item Update expected 1DP28BVO range
\item Changed formatting of ``Tlm Fmt'' in table
\item Changed time column from units of seconds to minutes in table
\item Changed text in table column ``Description''
\end{itemize}

\newpage\
\vspace{0.4\textheight}
\bc This page is intentionally blank \ec

\newcommand{\tablecaptiontext}{TURN ON DPA B (realtime version)}
\input{dpab_on.tab}

\end{document}


\end{document}


\end{document}


\end{document}
