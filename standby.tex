\documentclass[11pt]{article}

\usepackage{lscape,color}
\usepackage[normalem]{ulem}

\topmargin -0.75truein
\oddsidemargin -0.4truein
\textheight 9.25truein
\textwidth 6.7truein
\hbadness=10001
\hfuzz=200pt


\begin{document}
%\input dspace12.tex
%\input pstricks.tex
%\input psfig
\newcommand{\be}{\begin{enumerate}}
\newcommand{\ee}{\end{enumerate}}
\newcommand{\bc}{\begin{center}}
\newcommand{\ec}{\end{center}}
\newcommand{\bi}{\begin{itemize}}
\newcommand{\ei}{\end{itemize}}
\newcommand{\bd}{\begin{description}}
\newcommand{\ed}{\end{description}}
\newcommand{\bt}{\begin{tabbing}}
\newcommand{\et}{\end{tabbing}}
\newcommand{\eg}{{\it e.g.~}}
\newcommand{\ie}{{\it i.e.~}}
\newcommand{\ul}{\underline}
\newcommand{\axaf}{{\em AXAF}}
\newcommand*\red{\color{red}}
\newcommand*\blue{\color{blue}}
\def\la{\hbox{\rlap{$<$}\lower0.5ex\hbox{$\sim$}\ }}


\large
%\vspace*{-0.5in}
\centerline {\bf 4.11\_V2.1 PUT ACIS INTO THERMAL STANDBY MODE } 
\vspace{0.25in}

\normalsize
\noindent{\it Last Revised: April 7, 2017}\\
\noindent{\bf Filename: standby} \\


\noindent {\bf BRIEF FUNCTIONAL DESCRIPTION:} \\

This procedure is intended to put ACIS into a relatively low 
power state while preserving the flight SW patches in memory and 
active control of the focal plane temperature by maintaining power 
to at least one side of the DPA and one side of the DEA. The FEPs 
and the video boards will be turned off to reduce power consumption. 
ACIS should consume about 50~W in this configuration. The power 
breakdown is 12~W for DPA-A, 8~W for DPA-B, 24+/-4~W for DEA-A, 
4~W for the PSMC overhead, and 3~W for the FP heater. It is expected 
that the previous science run will have been halted by a ``Stop Science'' 
command included in the daily command load and scheduled at the appropriate 
time by the OFLS SW. Nevertheless, since it is no harm to the instrument 
to send this command twice, this command is sent again in this procedure. 
The procedure then turns off the video boards and FEPs but leaves everything 
else up and running. The advantage of this configuration is that the 
flight SW patches are preserved in memory and active control of the FP 
temperature is maintained such that it would be relatively quick to resume 
science operations.

\normalsize
\vspace{0.25in}
\noindent The sequence of actions will be: 
\be
\item issue a Stop Science Run command
\vspace{-0.10in}
\item turn off the video board power and FEPs, dump the system configuration
\vspace{-0.10in}
\item verify power consumption and Camera Body and Focal Plane temperatures
\ee


\vspace{0.15in}
\normalsize
\noindent {\bf ASSUMED INSTRUMENT STATE:} \\
\normalsize
Assumes that at least one side of the DPA and DEA A are on.\\
The instrument should not be in bakeout mode.\\

%\vspace{0.25in}
\normalsize
\noindent {\bf SPECIAL INITIAL CONDITIONS:} \\
\normalsize
None.
\\
%\newpage 

%\vspace{0.25in}
\normalsize
\noindent {\bf OPERATIONAL CONSTRAINTS/CAUTIONS:} \\
\normalsize
This procedure maintains the active thermal control of the instrument
so that the Camera Body and Focal Plane maintain their operational 
temperatures of $\sim$-60~C and $\sim$-120~C.\\

\vspace{0.15in}
\normalsize
\noindent {\bf CHANGE HISTORY:} \\
\normalsize

{\bf V1.2}
\begin{itemize}
\item changed command to power-down the FEPs and video boards to be
``WSPOW00000'' in step 2.1
\item modified Assumed Instrument State and Operational Constraints
and Cautions 
\end{itemize}

{\bf V2.0}
\begin{itemize}
\item ACIS Team signed-off version
\item added step 3 to verify DPA A \& B, DEA A and DA Htr B power
consumption and to verify Camera Body and FP temperature
\end{itemize}

{\bf V2.1}
\begin{itemize}
\item Removed verification steps for detector housing heater telemetry
\item Updated expected values for telemetry verifiers
\item Updated text to reflect current operational paradigm circa 2017
\end{itemize}

\newpage\
\vspace{0.4\textheight}
\bc This page is intentionally blank \ec

\noindent

\newcommand{\tablecaptiontext}{Put ACIS into Thermal Standby Mode~}
\documentclass[11pt]{article}

\usepackage{lscape,color}
\usepackage[normalem]{ulem}

\topmargin -0.75truein
\oddsidemargin -0.4truein
\textheight 9.25truein
\textwidth 6.7truein
\hbadness=10001
\hfuzz=200pt


\begin{document}
%\input dspace12.tex
%\input pstricks.tex
%\input psfig
\newcommand{\be}{\begin{enumerate}}
\newcommand{\ee}{\end{enumerate}}
\newcommand{\bc}{\begin{center}}
\newcommand{\ec}{\end{center}}
\newcommand{\bi}{\begin{itemize}}
\newcommand{\ei}{\end{itemize}}
\newcommand{\bd}{\begin{description}}
\newcommand{\ed}{\end{description}}
\newcommand{\bt}{\begin{tabbing}}
\newcommand{\et}{\end{tabbing}}
\newcommand{\eg}{{\it e.g.~}}
\newcommand{\ie}{{\it i.e.~}}
\newcommand{\ul}{\underline}
\newcommand{\axaf}{{\em AXAF}}
\newcommand*\red{\color{red}}
\newcommand*\blue{\color{blue}}
\def\la{\hbox{\rlap{$<$}\lower0.5ex\hbox{$\sim$}\ }}


\large
%\vspace*{-0.5in}
\centerline {\bf 4.11\_V2.1 PUT ACIS INTO THERMAL STANDBY MODE } 
\vspace{0.25in}

\normalsize
\noindent{\it Last Revised: April 7, 2017}\\
\noindent{\bf Filename: standby} \\


\noindent {\bf BRIEF FUNCTIONAL DESCRIPTION:} \\

This procedure is intended to put ACIS into a relatively low 
power state while preserving the flight SW patches in memory and 
active control of the focal plane temperature by maintaining power 
to at least one side of the DPA and one side of the DEA. The FEPs 
and the video boards will be turned off to reduce power consumption. 
ACIS should consume about 50~W in this configuration. The power 
breakdown is 12~W for DPA-A, 8~W for DPA-B, 24+/-4~W for DEA-A, 
4~W for the PSMC overhead, and 3~W for the FP heater. It is expected 
that the previous science run will have been halted by a ``Stop Science'' 
command included in the daily command load and scheduled at the appropriate 
time by the OFLS SW. Nevertheless, since it is no harm to the instrument 
to send this command twice, this command is sent again in this procedure. 
The procedure then turns off the video boards and FEPs but leaves everything 
else up and running. The advantage of this configuration is that the 
flight SW patches are preserved in memory and active control of the FP 
temperature is maintained such that it would be relatively quick to resume 
science operations.

\normalsize
\vspace{0.25in}
\noindent The sequence of actions will be: 
\be
\item issue a Stop Science Run command
\vspace{-0.10in}
\item turn off the video board power and FEPs, dump the system configuration
\vspace{-0.10in}
\item verify power consumption and Camera Body and Focal Plane temperatures
\ee


\vspace{0.15in}
\normalsize
\noindent {\bf ASSUMED INSTRUMENT STATE:} \\
\normalsize
Assumes that at least one side of the DPA and DEA A are on.\\
The instrument should not be in bakeout mode.\\

%\vspace{0.25in}
\normalsize
\noindent {\bf SPECIAL INITIAL CONDITIONS:} \\
\normalsize
None.
\\
%\newpage 

%\vspace{0.25in}
\normalsize
\noindent {\bf OPERATIONAL CONSTRAINTS/CAUTIONS:} \\
\normalsize
This procedure maintains the active thermal control of the instrument
so that the Camera Body and Focal Plane maintain their operational 
temperatures of $\sim$-60~C and $\sim$-120~C.\\

\vspace{0.15in}
\normalsize
\noindent {\bf CHANGE HISTORY:} \\
\normalsize

{\bf V1.2}
\begin{itemize}
\item changed command to power-down the FEPs and video boards to be
``WSPOW00000'' in step 2.1
\item modified Assumed Instrument State and Operational Constraints
and Cautions 
\end{itemize}

{\bf V2.0}
\begin{itemize}
\item ACIS Team signed-off version
\item added step 3 to verify DPA A \& B, DEA A and DA Htr B power
consumption and to verify Camera Body and FP temperature
\end{itemize}

{\bf V2.1}
\begin{itemize}
\item Removed verification steps for detector housing heater telemetry
\item Updated expected values for telemetry verifiers
\item Updated text to reflect current operational paradigm circa 2017
\end{itemize}

\newpage\
\vspace{0.4\textheight}
\bc This page is intentionally blank \ec

\noindent

\newcommand{\tablecaptiontext}{Put ACIS into Thermal Standby Mode~}
\documentclass[11pt]{article}

\usepackage{lscape,color}
\usepackage[normalem]{ulem}

\topmargin -0.75truein
\oddsidemargin -0.4truein
\textheight 9.25truein
\textwidth 6.7truein
\hbadness=10001
\hfuzz=200pt


\begin{document}
%\input dspace12.tex
%\input pstricks.tex
%\input psfig
\newcommand{\be}{\begin{enumerate}}
\newcommand{\ee}{\end{enumerate}}
\newcommand{\bc}{\begin{center}}
\newcommand{\ec}{\end{center}}
\newcommand{\bi}{\begin{itemize}}
\newcommand{\ei}{\end{itemize}}
\newcommand{\bd}{\begin{description}}
\newcommand{\ed}{\end{description}}
\newcommand{\bt}{\begin{tabbing}}
\newcommand{\et}{\end{tabbing}}
\newcommand{\eg}{{\it e.g.~}}
\newcommand{\ie}{{\it i.e.~}}
\newcommand{\ul}{\underline}
\newcommand{\axaf}{{\em AXAF}}
\newcommand*\red{\color{red}}
\newcommand*\blue{\color{blue}}
\def\la{\hbox{\rlap{$<$}\lower0.5ex\hbox{$\sim$}\ }}


\large
%\vspace*{-0.5in}
\centerline {\bf 4.11\_V2.1 PUT ACIS INTO THERMAL STANDBY MODE } 
\vspace{0.25in}

\normalsize
\noindent{\it Last Revised: April 7, 2017}\\
\noindent{\bf Filename: standby} \\


\noindent {\bf BRIEF FUNCTIONAL DESCRIPTION:} \\

This procedure is intended to put ACIS into a relatively low 
power state while preserving the flight SW patches in memory and 
active control of the focal plane temperature by maintaining power 
to at least one side of the DPA and one side of the DEA. The FEPs 
and the video boards will be turned off to reduce power consumption. 
ACIS should consume about 50~W in this configuration. The power 
breakdown is 12~W for DPA-A, 8~W for DPA-B, 24+/-4~W for DEA-A, 
4~W for the PSMC overhead, and 3~W for the FP heater. It is expected 
that the previous science run will have been halted by a ``Stop Science'' 
command included in the daily command load and scheduled at the appropriate 
time by the OFLS SW. Nevertheless, since it is no harm to the instrument 
to send this command twice, this command is sent again in this procedure. 
The procedure then turns off the video boards and FEPs but leaves everything 
else up and running. The advantage of this configuration is that the 
flight SW patches are preserved in memory and active control of the FP 
temperature is maintained such that it would be relatively quick to resume 
science operations.

\normalsize
\vspace{0.25in}
\noindent The sequence of actions will be: 
\be
\item issue a Stop Science Run command
\vspace{-0.10in}
\item turn off the video board power and FEPs, dump the system configuration
\vspace{-0.10in}
\item verify power consumption and Camera Body and Focal Plane temperatures
\ee


\vspace{0.15in}
\normalsize
\noindent {\bf ASSUMED INSTRUMENT STATE:} \\
\normalsize
Assumes that at least one side of the DPA and DEA A are on.\\
The instrument should not be in bakeout mode.\\

%\vspace{0.25in}
\normalsize
\noindent {\bf SPECIAL INITIAL CONDITIONS:} \\
\normalsize
None.
\\
%\newpage 

%\vspace{0.25in}
\normalsize
\noindent {\bf OPERATIONAL CONSTRAINTS/CAUTIONS:} \\
\normalsize
This procedure maintains the active thermal control of the instrument
so that the Camera Body and Focal Plane maintain their operational 
temperatures of $\sim$-60~C and $\sim$-120~C.\\

\vspace{0.15in}
\normalsize
\noindent {\bf CHANGE HISTORY:} \\
\normalsize

{\bf V1.2}
\begin{itemize}
\item changed command to power-down the FEPs and video boards to be
``WSPOW00000'' in step 2.1
\item modified Assumed Instrument State and Operational Constraints
and Cautions 
\end{itemize}

{\bf V2.0}
\begin{itemize}
\item ACIS Team signed-off version
\item added step 3 to verify DPA A \& B, DEA A and DA Htr B power
consumption and to verify Camera Body and FP temperature
\end{itemize}

{\bf V2.1}
\begin{itemize}
\item Removed verification steps for detector housing heater telemetry
\item Updated expected values for telemetry verifiers
\item Updated text to reflect current operational paradigm circa 2017
\end{itemize}

\newpage\
\vspace{0.4\textheight}
\bc This page is intentionally blank \ec

\noindent

\newcommand{\tablecaptiontext}{Put ACIS into Thermal Standby Mode~}
\documentclass[11pt]{article}

\usepackage{lscape,color}
\usepackage[normalem]{ulem}

\topmargin -0.75truein
\oddsidemargin -0.4truein
\textheight 9.25truein
\textwidth 6.7truein
\hbadness=10001
\hfuzz=200pt


\begin{document}
%\input dspace12.tex
%\input pstricks.tex
%\input psfig
\newcommand{\be}{\begin{enumerate}}
\newcommand{\ee}{\end{enumerate}}
\newcommand{\bc}{\begin{center}}
\newcommand{\ec}{\end{center}}
\newcommand{\bi}{\begin{itemize}}
\newcommand{\ei}{\end{itemize}}
\newcommand{\bd}{\begin{description}}
\newcommand{\ed}{\end{description}}
\newcommand{\bt}{\begin{tabbing}}
\newcommand{\et}{\end{tabbing}}
\newcommand{\eg}{{\it e.g.~}}
\newcommand{\ie}{{\it i.e.~}}
\newcommand{\ul}{\underline}
\newcommand{\axaf}{{\em AXAF}}
\newcommand*\red{\color{red}}
\newcommand*\blue{\color{blue}}
\def\la{\hbox{\rlap{$<$}\lower0.5ex\hbox{$\sim$}\ }}


\large
%\vspace*{-0.5in}
\centerline {\bf 4.11\_V2.1 PUT ACIS INTO THERMAL STANDBY MODE } 
\vspace{0.25in}

\normalsize
\noindent{\it Last Revised: April 7, 2017}\\
\noindent{\bf Filename: standby} \\


\noindent {\bf BRIEF FUNCTIONAL DESCRIPTION:} \\

This procedure is intended to put ACIS into a relatively low 
power state while preserving the flight SW patches in memory and 
active control of the focal plane temperature by maintaining power 
to at least one side of the DPA and one side of the DEA. The FEPs 
and the video boards will be turned off to reduce power consumption. 
ACIS should consume about 50~W in this configuration. The power 
breakdown is 12~W for DPA-A, 8~W for DPA-B, 24+/-4~W for DEA-A, 
4~W for the PSMC overhead, and 3~W for the FP heater. It is expected 
that the previous science run will have been halted by a ``Stop Science'' 
command included in the daily command load and scheduled at the appropriate 
time by the OFLS SW. Nevertheless, since it is no harm to the instrument 
to send this command twice, this command is sent again in this procedure. 
The procedure then turns off the video boards and FEPs but leaves everything 
else up and running. The advantage of this configuration is that the 
flight SW patches are preserved in memory and active control of the FP 
temperature is maintained such that it would be relatively quick to resume 
science operations.

\normalsize
\vspace{0.25in}
\noindent The sequence of actions will be: 
\be
\item issue a Stop Science Run command
\vspace{-0.10in}
\item turn off the video board power and FEPs, dump the system configuration
\vspace{-0.10in}
\item verify power consumption and Camera Body and Focal Plane temperatures
\ee


\vspace{0.15in}
\normalsize
\noindent {\bf ASSUMED INSTRUMENT STATE:} \\
\normalsize
Assumes that at least one side of the DPA and DEA A are on.\\
The instrument should not be in bakeout mode.\\

%\vspace{0.25in}
\normalsize
\noindent {\bf SPECIAL INITIAL CONDITIONS:} \\
\normalsize
None.
\\
%\newpage 

%\vspace{0.25in}
\normalsize
\noindent {\bf OPERATIONAL CONSTRAINTS/CAUTIONS:} \\
\normalsize
This procedure maintains the active thermal control of the instrument
so that the Camera Body and Focal Plane maintain their operational 
temperatures of $\sim$-60~C and $\sim$-120~C.\\

\vspace{0.15in}
\normalsize
\noindent {\bf CHANGE HISTORY:} \\
\normalsize

{\bf V1.2}
\begin{itemize}
\item changed command to power-down the FEPs and video boards to be
``WSPOW00000'' in step 2.1
\item modified Assumed Instrument State and Operational Constraints
and Cautions 
\end{itemize}

{\bf V2.0}
\begin{itemize}
\item ACIS Team signed-off version
\item added step 3 to verify DPA A \& B, DEA A and DA Htr B power
consumption and to verify Camera Body and FP temperature
\end{itemize}

{\bf V2.1}
\begin{itemize}
\item Removed verification steps for detector housing heater telemetry
\item Updated expected values for telemetry verifiers
\item Updated text to reflect current operational paradigm circa 2017
\end{itemize}

\newpage\
\vspace{0.4\textheight}
\bc This page is intentionally blank \ec

\noindent

\newcommand{\tablecaptiontext}{Put ACIS into Thermal Standby Mode~}
\input{standby.tab}

\end{document}


\end{document}


\end{document}


\end{document}
