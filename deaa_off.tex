\documentclass[11pt]{article}

\usepackage{lscape,color}

\topmargin -0.75truein
\oddsidemargin -0.4truein
\textheight 9.25truein
\textwidth 6.7truein
\hbadness=10001
\hfuzz=200pt


\begin{document}
%\input dspace12.tex
%\input pstricks.tex
%\input psfig
\newcommand{\be}{\begin{enumerate}}
\newcommand{\ee}{\end{enumerate}}
\newcommand{\bc}{\begin{center}}
\newcommand{\ec}{\end{center}}
\newcommand{\bi}{\begin{itemize}}
\newcommand{\ei}{\end{itemize}}
\newcommand{\bd}{\begin{description}}
\newcommand{\ed}{\end{description}}
\newcommand{\bt}{\begin{tabbing}}
\newcommand{\et}{\end{tabbing}}
\newcommand{\eg}{{\it e.g.~}}
\newcommand{\ie}{{\it i.e.~}}
\newcommand{\ul}{\underline}
\newcommand{\axaf}{{\em AXAF}}
\def\la{\hbox{\rlap{$<$}\lower0.5ex\hbox{$\sim$}\ }}


\large
%\vspace*{-0.5in}
\centerline {\bf 4.8\_V2.4 TURN OFF DEA A}
\vspace{0.25in}

\normalsize
\noindent{\it Last Revised: July 12, 2017}\\
\noindent{\bf Filename: deaa\_off} \\

\noindent {\bf BRIEF FUNCTIONAL DESCRIPTION:} \\
\normalsize
This is an ``atomic'' procedure which simply turns off the DEA side A.
It should be safe to execute under any condition except a spacecraft 
power or thermal emergency.

\vspace{0.25in}
\noindent The sequence of actions for this procedure will be:
\be
\item Power down all the video boards
\vspace{-0.10in}
\item Turn off and disable DEA power supply side A
\ee

\vspace{0.15in}
\normalsize
\noindent {\bf ASSUMED INSTRUMENT STATE:}
\normalsize
\be
\item Assumes that the PSMC has power from the spacecraft.
\vspace{-0.10in}
\item The instrument should not be executing a science run.
\vspace{-0.10in}
\item The instrument should not be in bakeout mode.
\vspace{-0.10in}
\item Assumes that one side of the DPA is powered and that 
a BEP is active and ready to execute commands. If not, skip 
Step 1, which issues the WSVIDALLDN command to the BEP.
\ee
\vspace{0.1in}
\normalsize
\noindent {\bf SPECIAL INITIAL CONDITIONS:} \\
\normalsize
%The environment must be clean enough to allow opening of the valve. \\
%Assumes that \axaf\/ ISIM RCTU is powered on and in telemetry format 6. \\

%\newpage

%\vspace{0.25in}
\normalsize
\noindent {\bf OPERATIONAL CONSTRAINTS/CAUTIONS:} \\
\normalsize

The DEA power status is normally indicated by the values of the 1DEPSA and
1DEPSB flags, which should not both be 1 simultaneously.
However, if neither side of the DPA is receiving power
({\it i.e.}, if 1DPP0AVO and 1DPP0BVO are simultaneously reading $0.0 \pm 0.5$ V),
the DEA flag values will be unreliable and the DEA voltage
channels (1DEP[0123][AB]VO) should instead be used to determine which
sides of the DEA are powered).

If neither side of the DPA is receiving power, then the command to power down the 
video boards should be skipped.

Typically 1DEICACU (and likewise the B-side 1DEICBCU) reads 16-−18~A when 
unpowered, as of Telemetry Database (TDB) v14. This is expected and not a problem.

The instrument should not be executing a science run using DEA side A. 
If power is removed from DEA side A while a science run is in progress, the science 
run will terminate producing errors. The data will still be useful, but the processing 
will be more complicated.

The instrument should not be in bakeout mode. If the DEA side A power is 
removed during a bakeout, the FP bakeout heater will lose power and the bakeout of 
the FP will terminate. The DH bakeout heater is unaffected by a power loss to the 
DEA and will therefore still be executing a bakeout if power is lost to the DEA.

\vspace{0.15in}
\normalsize
\noindent {\bf REFERENCES:} \\
\normalsize

\normalsize
\noindent {\bf CHANGE HISTORY:} \\
\normalsize

{\bf V1.2}
\begin{itemize}
\item changed filenames from ``turnoff\_deaa'' to
``deaa\_off''
\item added text to explain the confusion with the logical verifiers
\end{itemize}

{\bf V1.3}
\begin{itemize}
\item changed primary verifier to be the DEA +24~V supply
\item changed TLM FMT to 1,2,4or6
\item changed expected value of DEA A Input Voltage to 28.0--34.0~V
\end{itemize}

{\bf V2.0}
\begin{itemize}
\item ACIS Team signed-off version, identical to previous version 1.3
\end{itemize}

{\bf V2.1}
\begin{itemize}
\item Update expected 1DE28AVO range
\item Changed formatting of ``Tlm Fmt'' in table
\item Changed time column from units of seconds to minutes in table
\item Changed text in table column ``Description''
\end{itemize}

{\bf V2.2}
\begin{itemize}
\item Removed check of 1DEICACU
\item Removed reference to 0~W of power from table
\item Reordered columns to match that on real-time webpages
\item Added error bars to voltages
\end{itemize}

{\bf V2.3}
\begin{itemize}
\item Moved the text regarding power status issues and expected current behavior 
from the Functional Description to the Operational Constraints/Cautions section 
and expanded it.
\item Added a step to turn off the video boards to the beginning of the procedure
\end{itemize}

{\bf V2.4}
\begin{itemize}
\item Re-arranged the power verifers so that they occur in between the command
to power the DEA off and the command to disable it. 
\item Added language stating that if the DPA is unpowered the WSVIDALLDN command
should not be issued.
\end{itemize}

\newpage\
\vspace{0.4\textheight}
\bc This page is intentionally blank \ec

\newcommand{\tablecaptiontext}{TURN OFF DEA A}
\documentclass[11pt]{article}

\usepackage{lscape,color}

\topmargin -0.75truein
\oddsidemargin -0.4truein
\textheight 9.25truein
\textwidth 6.7truein
\hbadness=10001
\hfuzz=200pt


\begin{document}
%\input dspace12.tex
%\input pstricks.tex
%\input psfig
\newcommand{\be}{\begin{enumerate}}
\newcommand{\ee}{\end{enumerate}}
\newcommand{\bc}{\begin{center}}
\newcommand{\ec}{\end{center}}
\newcommand{\bi}{\begin{itemize}}
\newcommand{\ei}{\end{itemize}}
\newcommand{\bd}{\begin{description}}
\newcommand{\ed}{\end{description}}
\newcommand{\bt}{\begin{tabbing}}
\newcommand{\et}{\end{tabbing}}
\newcommand{\eg}{{\it e.g.~}}
\newcommand{\ie}{{\it i.e.~}}
\newcommand{\ul}{\underline}
\newcommand{\axaf}{{\em AXAF}}
\def\la{\hbox{\rlap{$<$}\lower0.5ex\hbox{$\sim$}\ }}


\large
%\vspace*{-0.5in}
\centerline {\bf 4.8\_V2.4 TURN OFF DEA A}
\vspace{0.25in}

\normalsize
\noindent{\it Last Revised: July 12, 2017}\\
\noindent{\bf Filename: deaa\_off} \\

\noindent {\bf BRIEF FUNCTIONAL DESCRIPTION:} \\
\normalsize
This is an ``atomic'' procedure which simply turns off the DEA side A.
It should be safe to execute under any condition except a spacecraft 
power or thermal emergency.

\vspace{0.25in}
\noindent The sequence of actions for this procedure will be:
\be
\item Power down all the video boards
\vspace{-0.10in}
\item Turn off and disable DEA power supply side A
\ee

\vspace{0.15in}
\normalsize
\noindent {\bf ASSUMED INSTRUMENT STATE:}
\normalsize
\be
\item Assumes that the PSMC has power from the spacecraft.
\vspace{-0.10in}
\item The instrument should not be executing a science run.
\vspace{-0.10in}
\item The instrument should not be in bakeout mode.
\vspace{-0.10in}
\item Assumes that one side of the DPA is powered and that 
a BEP is active and ready to execute commands. If not, skip 
Step 1, which issues the WSVIDALLDN command to the BEP.
\ee
\vspace{0.1in}
\normalsize
\noindent {\bf SPECIAL INITIAL CONDITIONS:} \\
\normalsize
%The environment must be clean enough to allow opening of the valve. \\
%Assumes that \axaf\/ ISIM RCTU is powered on and in telemetry format 6. \\

%\newpage

%\vspace{0.25in}
\normalsize
\noindent {\bf OPERATIONAL CONSTRAINTS/CAUTIONS:} \\
\normalsize

The DEA power status is normally indicated by the values of the 1DEPSA and
1DEPSB flags, which should not both be 1 simultaneously.
However, if neither side of the DPA is receiving power
({\it i.e.}, if 1DPP0AVO and 1DPP0BVO are simultaneously reading $0.0 \pm 0.5$ V),
the DEA flag values will be unreliable and the DEA voltage
channels (1DEP[0123][AB]VO) should instead be used to determine which
sides of the DEA are powered).

If neither side of the DPA is receiving power, then the command to power down the 
video boards should be skipped.

Typically 1DEICACU (and likewise the B-side 1DEICBCU) reads 16-−18~A when 
unpowered, as of Telemetry Database (TDB) v14. This is expected and not a problem.

The instrument should not be executing a science run using DEA side A. 
If power is removed from DEA side A while a science run is in progress, the science 
run will terminate producing errors. The data will still be useful, but the processing 
will be more complicated.

The instrument should not be in bakeout mode. If the DEA side A power is 
removed during a bakeout, the FP bakeout heater will lose power and the bakeout of 
the FP will terminate. The DH bakeout heater is unaffected by a power loss to the 
DEA and will therefore still be executing a bakeout if power is lost to the DEA.

\vspace{0.15in}
\normalsize
\noindent {\bf REFERENCES:} \\
\normalsize

\normalsize
\noindent {\bf CHANGE HISTORY:} \\
\normalsize

{\bf V1.2}
\begin{itemize}
\item changed filenames from ``turnoff\_deaa'' to
``deaa\_off''
\item added text to explain the confusion with the logical verifiers
\end{itemize}

{\bf V1.3}
\begin{itemize}
\item changed primary verifier to be the DEA +24~V supply
\item changed TLM FMT to 1,2,4or6
\item changed expected value of DEA A Input Voltage to 28.0--34.0~V
\end{itemize}

{\bf V2.0}
\begin{itemize}
\item ACIS Team signed-off version, identical to previous version 1.3
\end{itemize}

{\bf V2.1}
\begin{itemize}
\item Update expected 1DE28AVO range
\item Changed formatting of ``Tlm Fmt'' in table
\item Changed time column from units of seconds to minutes in table
\item Changed text in table column ``Description''
\end{itemize}

{\bf V2.2}
\begin{itemize}
\item Removed check of 1DEICACU
\item Removed reference to 0~W of power from table
\item Reordered columns to match that on real-time webpages
\item Added error bars to voltages
\end{itemize}

{\bf V2.3}
\begin{itemize}
\item Moved the text regarding power status issues and expected current behavior 
from the Functional Description to the Operational Constraints/Cautions section 
and expanded it.
\item Added a step to turn off the video boards to the beginning of the procedure
\end{itemize}

{\bf V2.4}
\begin{itemize}
\item Re-arranged the power verifers so that they occur in between the command
to power the DEA off and the command to disable it. 
\item Added language stating that if the DPA is unpowered the WSVIDALLDN command
should not be issued.
\end{itemize}

\newpage\
\vspace{0.4\textheight}
\bc This page is intentionally blank \ec

\newcommand{\tablecaptiontext}{TURN OFF DEA A}
\documentclass[11pt]{article}

\usepackage{lscape,color}

\topmargin -0.75truein
\oddsidemargin -0.4truein
\textheight 9.25truein
\textwidth 6.7truein
\hbadness=10001
\hfuzz=200pt


\begin{document}
%\input dspace12.tex
%\input pstricks.tex
%\input psfig
\newcommand{\be}{\begin{enumerate}}
\newcommand{\ee}{\end{enumerate}}
\newcommand{\bc}{\begin{center}}
\newcommand{\ec}{\end{center}}
\newcommand{\bi}{\begin{itemize}}
\newcommand{\ei}{\end{itemize}}
\newcommand{\bd}{\begin{description}}
\newcommand{\ed}{\end{description}}
\newcommand{\bt}{\begin{tabbing}}
\newcommand{\et}{\end{tabbing}}
\newcommand{\eg}{{\it e.g.~}}
\newcommand{\ie}{{\it i.e.~}}
\newcommand{\ul}{\underline}
\newcommand{\axaf}{{\em AXAF}}
\def\la{\hbox{\rlap{$<$}\lower0.5ex\hbox{$\sim$}\ }}


\large
%\vspace*{-0.5in}
\centerline {\bf 4.8\_V2.4 TURN OFF DEA A}
\vspace{0.25in}

\normalsize
\noindent{\it Last Revised: July 12, 2017}\\
\noindent{\bf Filename: deaa\_off} \\

\noindent {\bf BRIEF FUNCTIONAL DESCRIPTION:} \\
\normalsize
This is an ``atomic'' procedure which simply turns off the DEA side A.
It should be safe to execute under any condition except a spacecraft 
power or thermal emergency.

\vspace{0.25in}
\noindent The sequence of actions for this procedure will be:
\be
\item Power down all the video boards
\vspace{-0.10in}
\item Turn off and disable DEA power supply side A
\ee

\vspace{0.15in}
\normalsize
\noindent {\bf ASSUMED INSTRUMENT STATE:}
\normalsize
\be
\item Assumes that the PSMC has power from the spacecraft.
\vspace{-0.10in}
\item The instrument should not be executing a science run.
\vspace{-0.10in}
\item The instrument should not be in bakeout mode.
\vspace{-0.10in}
\item Assumes that one side of the DPA is powered and that 
a BEP is active and ready to execute commands. If not, skip 
Step 1, which issues the WSVIDALLDN command to the BEP.
\ee
\vspace{0.1in}
\normalsize
\noindent {\bf SPECIAL INITIAL CONDITIONS:} \\
\normalsize
%The environment must be clean enough to allow opening of the valve. \\
%Assumes that \axaf\/ ISIM RCTU is powered on and in telemetry format 6. \\

%\newpage

%\vspace{0.25in}
\normalsize
\noindent {\bf OPERATIONAL CONSTRAINTS/CAUTIONS:} \\
\normalsize

The DEA power status is normally indicated by the values of the 1DEPSA and
1DEPSB flags, which should not both be 1 simultaneously.
However, if neither side of the DPA is receiving power
({\it i.e.}, if 1DPP0AVO and 1DPP0BVO are simultaneously reading $0.0 \pm 0.5$ V),
the DEA flag values will be unreliable and the DEA voltage
channels (1DEP[0123][AB]VO) should instead be used to determine which
sides of the DEA are powered).

If neither side of the DPA is receiving power, then the command to power down the 
video boards should be skipped.

Typically 1DEICACU (and likewise the B-side 1DEICBCU) reads 16-−18~A when 
unpowered, as of Telemetry Database (TDB) v14. This is expected and not a problem.

The instrument should not be executing a science run using DEA side A. 
If power is removed from DEA side A while a science run is in progress, the science 
run will terminate producing errors. The data will still be useful, but the processing 
will be more complicated.

The instrument should not be in bakeout mode. If the DEA side A power is 
removed during a bakeout, the FP bakeout heater will lose power and the bakeout of 
the FP will terminate. The DH bakeout heater is unaffected by a power loss to the 
DEA and will therefore still be executing a bakeout if power is lost to the DEA.

\vspace{0.15in}
\normalsize
\noindent {\bf REFERENCES:} \\
\normalsize

\normalsize
\noindent {\bf CHANGE HISTORY:} \\
\normalsize

{\bf V1.2}
\begin{itemize}
\item changed filenames from ``turnoff\_deaa'' to
``deaa\_off''
\item added text to explain the confusion with the logical verifiers
\end{itemize}

{\bf V1.3}
\begin{itemize}
\item changed primary verifier to be the DEA +24~V supply
\item changed TLM FMT to 1,2,4or6
\item changed expected value of DEA A Input Voltage to 28.0--34.0~V
\end{itemize}

{\bf V2.0}
\begin{itemize}
\item ACIS Team signed-off version, identical to previous version 1.3
\end{itemize}

{\bf V2.1}
\begin{itemize}
\item Update expected 1DE28AVO range
\item Changed formatting of ``Tlm Fmt'' in table
\item Changed time column from units of seconds to minutes in table
\item Changed text in table column ``Description''
\end{itemize}

{\bf V2.2}
\begin{itemize}
\item Removed check of 1DEICACU
\item Removed reference to 0~W of power from table
\item Reordered columns to match that on real-time webpages
\item Added error bars to voltages
\end{itemize}

{\bf V2.3}
\begin{itemize}
\item Moved the text regarding power status issues and expected current behavior 
from the Functional Description to the Operational Constraints/Cautions section 
and expanded it.
\item Added a step to turn off the video boards to the beginning of the procedure
\end{itemize}

{\bf V2.4}
\begin{itemize}
\item Re-arranged the power verifers so that they occur in between the command
to power the DEA off and the command to disable it. 
\item Added language stating that if the DPA is unpowered the WSVIDALLDN command
should not be issued.
\end{itemize}

\newpage\
\vspace{0.4\textheight}
\bc This page is intentionally blank \ec

\newcommand{\tablecaptiontext}{TURN OFF DEA A}
\documentclass[11pt]{article}

\usepackage{lscape,color}

\topmargin -0.75truein
\oddsidemargin -0.4truein
\textheight 9.25truein
\textwidth 6.7truein
\hbadness=10001
\hfuzz=200pt


\begin{document}
%\input dspace12.tex
%\input pstricks.tex
%\input psfig
\newcommand{\be}{\begin{enumerate}}
\newcommand{\ee}{\end{enumerate}}
\newcommand{\bc}{\begin{center}}
\newcommand{\ec}{\end{center}}
\newcommand{\bi}{\begin{itemize}}
\newcommand{\ei}{\end{itemize}}
\newcommand{\bd}{\begin{description}}
\newcommand{\ed}{\end{description}}
\newcommand{\bt}{\begin{tabbing}}
\newcommand{\et}{\end{tabbing}}
\newcommand{\eg}{{\it e.g.~}}
\newcommand{\ie}{{\it i.e.~}}
\newcommand{\ul}{\underline}
\newcommand{\axaf}{{\em AXAF}}
\def\la{\hbox{\rlap{$<$}\lower0.5ex\hbox{$\sim$}\ }}


\large
%\vspace*{-0.5in}
\centerline {\bf 4.8\_V2.4 TURN OFF DEA A}
\vspace{0.25in}

\normalsize
\noindent{\it Last Revised: July 12, 2017}\\
\noindent{\bf Filename: deaa\_off} \\

\noindent {\bf BRIEF FUNCTIONAL DESCRIPTION:} \\
\normalsize
This is an ``atomic'' procedure which simply turns off the DEA side A.
It should be safe to execute under any condition except a spacecraft 
power or thermal emergency.

\vspace{0.25in}
\noindent The sequence of actions for this procedure will be:
\be
\item Power down all the video boards
\vspace{-0.10in}
\item Turn off and disable DEA power supply side A
\ee

\vspace{0.15in}
\normalsize
\noindent {\bf ASSUMED INSTRUMENT STATE:}
\normalsize
\be
\item Assumes that the PSMC has power from the spacecraft.
\vspace{-0.10in}
\item The instrument should not be executing a science run.
\vspace{-0.10in}
\item The instrument should not be in bakeout mode.
\vspace{-0.10in}
\item Assumes that one side of the DPA is powered and that 
a BEP is active and ready to execute commands. If not, skip 
Step 1, which issues the WSVIDALLDN command to the BEP.
\ee
\vspace{0.1in}
\normalsize
\noindent {\bf SPECIAL INITIAL CONDITIONS:} \\
\normalsize
%The environment must be clean enough to allow opening of the valve. \\
%Assumes that \axaf\/ ISIM RCTU is powered on and in telemetry format 6. \\

%\newpage

%\vspace{0.25in}
\normalsize
\noindent {\bf OPERATIONAL CONSTRAINTS/CAUTIONS:} \\
\normalsize

The DEA power status is normally indicated by the values of the 1DEPSA and
1DEPSB flags, which should not both be 1 simultaneously.
However, if neither side of the DPA is receiving power
({\it i.e.}, if 1DPP0AVO and 1DPP0BVO are simultaneously reading $0.0 \pm 0.5$ V),
the DEA flag values will be unreliable and the DEA voltage
channels (1DEP[0123][AB]VO) should instead be used to determine which
sides of the DEA are powered).

If neither side of the DPA is receiving power, then the command to power down the 
video boards should be skipped.

Typically 1DEICACU (and likewise the B-side 1DEICBCU) reads 16-−18~A when 
unpowered, as of Telemetry Database (TDB) v14. This is expected and not a problem.

The instrument should not be executing a science run using DEA side A. 
If power is removed from DEA side A while a science run is in progress, the science 
run will terminate producing errors. The data will still be useful, but the processing 
will be more complicated.

The instrument should not be in bakeout mode. If the DEA side A power is 
removed during a bakeout, the FP bakeout heater will lose power and the bakeout of 
the FP will terminate. The DH bakeout heater is unaffected by a power loss to the 
DEA and will therefore still be executing a bakeout if power is lost to the DEA.

\vspace{0.15in}
\normalsize
\noindent {\bf REFERENCES:} \\
\normalsize

\normalsize
\noindent {\bf CHANGE HISTORY:} \\
\normalsize

{\bf V1.2}
\begin{itemize}
\item changed filenames from ``turnoff\_deaa'' to
``deaa\_off''
\item added text to explain the confusion with the logical verifiers
\end{itemize}

{\bf V1.3}
\begin{itemize}
\item changed primary verifier to be the DEA +24~V supply
\item changed TLM FMT to 1,2,4or6
\item changed expected value of DEA A Input Voltage to 28.0--34.0~V
\end{itemize}

{\bf V2.0}
\begin{itemize}
\item ACIS Team signed-off version, identical to previous version 1.3
\end{itemize}

{\bf V2.1}
\begin{itemize}
\item Update expected 1DE28AVO range
\item Changed formatting of ``Tlm Fmt'' in table
\item Changed time column from units of seconds to minutes in table
\item Changed text in table column ``Description''
\end{itemize}

{\bf V2.2}
\begin{itemize}
\item Removed check of 1DEICACU
\item Removed reference to 0~W of power from table
\item Reordered columns to match that on real-time webpages
\item Added error bars to voltages
\end{itemize}

{\bf V2.3}
\begin{itemize}
\item Moved the text regarding power status issues and expected current behavior 
from the Functional Description to the Operational Constraints/Cautions section 
and expanded it.
\item Added a step to turn off the video boards to the beginning of the procedure
\end{itemize}

{\bf V2.4}
\begin{itemize}
\item Re-arranged the power verifers so that they occur in between the command
to power the DEA off and the command to disable it. 
\item Added language stating that if the DPA is unpowered the WSVIDALLDN command
should not be issued.
\end{itemize}

\newpage\
\vspace{0.4\textheight}
\bc This page is intentionally blank \ec

\newcommand{\tablecaptiontext}{TURN OFF DEA A}
\input{deaa_off.tab}

\end{document}


\end{document}


\end{document}


\end{document}
